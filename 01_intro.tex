\section{Introduction}
\label{sec:intro}

Nowadays, the age of the Internet has changed the way people express their opinions and views.  It is now mainly done through blog posts, online forums, product review websites, social media, etc. Millions of people are using social network sites like Facebook, Twitter, Google Plus, etc. to express their feelings, opinion and share views about their daily lives. Social media is generating a massive amount of sentiment data in the form of tweets, status updates, blog posts, comments, reviews, etc.  Therefore, It is provided an opportunity for businesses by giving a platform to connect with their customers for advertising. People mostly depend on user-generated content online to a large extent to make decisions. For example, if someone wants to buy a product or wants to use any service, they firstly search for its reviews online, discuss it on social media before making a decision. The amount of content generated by users is too large for the normal user to analyze. Therefore, it is necessary to automate this, various sentiment analysis techniques are widely used. Sentiment analysis influences users to classify whether the information about the product is satisfactory or not before they acquire it. Marketers and firms use this analysis to understand their products or services in such a way that it can be offered as per the user’s needs.

There are two types of machine learning techniques that are generally used for sentiment analysis, one is unsupervised and the other is supervised. Unsupervised learning does not consist of a category and they do not provide with the correct targets at all and therefore conduct clustering. Supervised learning is based on the labeled dataset and thus the labels are provided to the model during the process. These labeled datasets are trained to produce reasonable outputs when encountered during decision- making. 

To help us to understand the sentiment analysis in a better way, this project is based on supervised machine learning. The rest of the paper is organized as follows. The second section discusses in brief about the work carried out for sentiment analysis in the different domain by various researchers. The third section is about the approach we followed for sentiment analysis. Section four is about implementation details and results followed by a conclusion and future work discussion in the last section. 
