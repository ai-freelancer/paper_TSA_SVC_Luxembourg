%%%%%%%%%%%%%%%%%%%%%%%%%%%%%%%%%%%%%%%%%%%%%%%%%%%%%%%%%%%%%%%%%%%%%%
%     File: ExtendedAbstract_abstr.tex                               %
%     Tex Master: ExtendedAbstract.tex                               %
%%
%% Abstract
%%
\begin{abstract}
Sentiment analysis refers to the use of machine learning, natural language processing, and computational linguistics to determine the sentiment content from the written language, i.e. it analyzes people's opinions, attitudes, and emotions for products, services, organizations, individuals, events or topics. In this project, we try to build the automatic sentiment analyzer system based on SVM and other techniques. The frequency of occurrence of words used as features for SVM. To evaluate our system, the Twitter dataset which containing tweets that are manually annotated for the sentiment (positive, negative or neutral) will be used. Eventually, we use the system to classify the unseen message's sentiments. \\

%%
%% Keywords (max 5)
%%
\noindent{{\bf Keywords:}} Machine Learning, Sentiment Analysis, SVMs, Feature Selection, Twitter Dataset, Nature Language Processing \\

\end{abstract}

