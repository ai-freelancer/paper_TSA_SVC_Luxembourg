\section{Conclusions}
\label{sec:concl}

The SVM has been widely used and promoted for land cover classification studies. So in this project, we proposed the used of SVM with sentiment analysis for classifying the sentence based on twitter data. There are certain issues while dealing with misspellings and slang words from tweets. To deal with these issues, the efficient feature vector is created by doing preprocessing. After preprocessing the raw data, we aim to build an automatic sentiment system by using the twitter dataset which is already labeled. The classification accuracy of the model is tested using the SVM classifier but with a different configuration of parameters. Unigram model is deployed with SVM to classify the sentiment of Twitter data. The training data set can be increased to improve the feature vector related sentence identification process. It may give a better visualization of the content in a better manner that will be helpful for the users. A major conclusion of this study is that the accuracy of SVM classification is influenced by the number of features used.


