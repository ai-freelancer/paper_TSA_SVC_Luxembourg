% A Theory section should extend, not repeat, the background to the
% article already dealt with in the Introduction and lay the
% foundation for further work.

\section{Related Work}
\label{sec: backg}

Work in the field "Sentiment analysis" has started since the beginning of the century. In its early stage, it was intended for binary classification, which assigns opinions or reviews to bipolar classes such as positive or negative. Paper \cite{turney2002thumbs} predicts review by the average semantic orientation of a phrase that contains adjectives and adverbs thus calculating whether the phrase is positive or negative with the use of unsupervised learning algorithm which classifies it as thumbs up or thumbs down the review. 
 
 In \cite{liu2011movie} the product feature uses a latent semantic analysis (LSA) based filtering mechanism to identify opinion words that are used to select some sentences to become a rich review summarization.  In another work \cite{khan2011sentiment}, the polarity of the word is being calculated by all the words in the sentence, which can either be positive or negative depending on the related sentence structure. In \cite{ramachandran2011automated} has proposed to pre-process the data to improve the quality structure of the raw sentence. They have applied the LSA technique and cosine similarity for sentiment analysis. \cite{karamibekr2012verb} proposed a method based on verbs as an important opinion term for sentiment classification of a document belonging to the social domain. In \cite{agarwal2013sentiment} applied phrase pattern method for sentiment classification. It uses part of speech based rules and dependency relation for extracting contextual and syntactic information from the document. 

Overall, a lot of work has also been done where researchers have explored and applied soft-computing approaches, mainly machine learning and neural works for sentiment analysis. In \cite{go2009twitter}, the authors introduce a novel approach for automatically classifying the sentiment of Twitter messages. Their main idea is using tweets with emoticons for distant supervised learning. They apply several machine learning algorithms like Naive Bayes, Maximum Entropy, and SVM. In \cite{gautam2014sentiment}, the authors applied several ML techniques like Naive Bayes, Maximum entropy and SVM along with the Semantic Orientation based WordNet which extracts synonyms and similarity for the content feature to analyze customer's review sentiment. 










